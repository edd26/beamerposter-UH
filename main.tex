\documentclass[final]{beamer}
\mode<presentation> {
  \usetheme{UH} 
}
\usepackage{subcaption}
\usepackage{lipsum}
\captionsetup{font=normalsize,labelfont={color=blue,normalsize},compatibility=false}
\usepackage{type1cm}
\usepackage[T1]{fontenc}
\usepackage{amsmath,amsthm, amssymb, latexsym}
\usepackage{multirow}
\usepackage{roboto}
\boldmath
\usepackage[orientation=portrait,size=a0,debug,scale=1.3]{beamerposter}


\bibliographystyle{IEEEtran}
% Reduce bibliography spacing
% https://www.math.cmu.edu/~gautam/sj/blog/20140712-bibtex-spacing.html
% \usepackage{bibspacing}
% \setlength{\bibitemsep}{.2\baselineskip plus .05\baselineskip minus .05\baselineskip}

\title[UH poster]{P52: High-dimensional topological analysis of BOLD sliding window correlations}

\author[Dmitruk]{Emil Dmitruk\textsuperscript{1}, Christoph Metzner\textsuperscript{1,2,3}, Volker Steuber\textsuperscript{1} and Shabnam Kadir\textsuperscript{1}
}

% \author[Dmitruk]{Emil Dmitruk\textsuperscript{1}}
% \author[Metzner]{Christoph Metzner\textsuperscript{2}}
% \author[Steuber]{Volker Steuber\textsuperscript{1}}
% \author[Kadir]{Shabnam Kadir\textsuperscript{1}}


\institute[UH]{\textsuperscript{1}University of Hertfordshire, UK; \textsuperscript{2}Technische Universität Berlin, Germany;\textsuperscript{3}Charité-Universitätsmedizin Berlin, Germany}

\begin{document}
\begin{frame}{} 
  \begin{block}{Introduction}
  \small{
    % \begin{center}
      In recent years, topological insights into neuroscience have brought more understanding about the functional and structural connectivity of the brain \cite{Saggar2018a}\cite{Sizemore2018}. There is evidence that topological properties of brain connectomics are different for schizophrenia subjects and healthy controls \cite{Stolz2018} \cite{Dmitruk2021} but also that the topology can be used to reveal the brain state trajectories \cite{Rieck2020a}. Inspired by those studies we aim to further investigate the brain dynamics and turn our attention to the dynamics of brain pathological states.
      
      
    In this work we aim at establishing the framework for analysis of brain functional dynamics by studying the evolution of its higher-dimensional topological properties.
    We use Midnight Scan Club's~\cite{Gordon2017} fMRI recordings acquired during motor task execution.
    }
    % \end{center}
  \end{block}
  % ===-===-===-===-===-===-===-===-===-===-
\begin{columns}
    \begin{column}{0.65\textwidth}
        \begin{block}{1. Data and pre-processing}
            \begin{figure}[H]
                \centering
                \includegraphics[width=0.9\linewidth]{images/bernstein2022_abstract_top.pdf}
                \label{fig:preproc}
            \end{figure}
            \centering{
            \small{
            \begin{columns}
                \begin{column}{0.49\textwidth}
                    Preprocessing:
                    \begin{itemize}
                        \small{
                        \item Brain extraction
                        \item Spatial smoothing
                        \item ICA-AROMA~\cite{Pruim2015} noise removal
                        \item AAL2~\cite{Rolls2015} motor cortex signal extraction
                    }
                    \end{itemize}
                \end{column}
                \begin{column}{0.45\textwidth}
                    Topology of BOLD sliding window correlations
                    \begin{itemize}
                        \small{
                        \item Correlation window size: 11 seconds
                        \item Compute the order complex (a simplicial complex) from the matrix of correlations, and obtain persistent homology 
                        \item Obtain pairwise Wasserstein distances for birth-death diagrams corresponding to time points
                        }
                    \end{itemize}
                \end{column}
            \end{columns}
            } % small
            } % centering
        
        \end{block}
    \end{column}
    
    \begin{column}{0.34\textwidth}
        \begin{block}{2. Topological characteristics of dynamic brain states}
            \begin{figure}[H]
                \centering
                \includegraphics[width=\linewidth]{images/2-wmat.pdf}
                \label{fig:preproc}
            \end{figure}
        \end{block}
    \end{column} %sec 2 and 3
\end{columns}


    % ===-===-===-===-===-===-===-===-===-===-  
    \begin{block}{3. Differences in topological characteristic distributions}
    \begin{columns}
            \begin{column}{0.25\textwidth}
                \begin{figure}[H]
                    \centering
                    \includegraphics[width=\linewidth]{images/matrix_comparison.pdf}
                    \label{fig:distros}
                \end{figure}
                \begin{itemize}
                \small{
                    \item MSC: distribution of Wasserstein distance matrix computed for a motor task, voxels from motor cortex
                    \item Shuffle: time series shuffling of fMRI recording
                    \item Geometric: pairwise Euclidean distance of samples from n-dimensional unit cube
                    }
                \end{itemize}
                % }
            \end{column}
            \begin{column}{0.70\textwidth}
                \begin{figure}[H]
                    \centering
                      \includegraphics[width=\linewidth]{images/distributions.pdf}
                    \label{fig:distros}
                \end{figure}
            \end{column}
            
    \end{columns}
    \end{block}
    
    \begin{block}{4. Summary}
    \begin{columns}
            \begin{column}{0.48\textwidth}
            
                \textbf{Results:}
                \begin{itemize}
                \small{
                    \item Some of the task transitions are correlated in time with the occurrence of higher-ordered topological structures
                    \item We hypothesise that those high-dimensional topological structures are related to the brain state transitions
                    \item  Not all transitions are accompanied by immediate formation of those structures 
                    \item The consecutive simplicial complexes are strikingly different
                }
                \end{itemize}
            
            \end{column}    
            \begin{column}{0.48\textwidth}
                \textbf{Conclusions:}
                \begin{itemize}
                \small{
                    \item First application of high-dimensional topology in the study of dynamics of brain activity
                    \item This topological method enables analysis of the functional connectomics of individuals using just a single recording, unlike most standard pipelines for the analysis of fMRI data.
                    \item Better understanding of the differences between individuals’ brain functional connectivities
                }
                \end{itemize}
            
            \end{column}
        \end{columns}
    \end{block}
    
    \begin{block}{5. References}
    \tiny{
        \bibliography{biblio2022}
    }
    \end{block}
%     \end{column}
%   \end{columns}
  
  
  
\end{frame}
\end{document}
